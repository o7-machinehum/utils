\documentclass[9pt]{developercv} % Default font size, values from 8-12pt are recommended
\begin{document}

%----------------------------------------------------------------------------------------
%	TITLE AND CONTACT INFORMATION
%----------------------------------------------------------------------------------------

\begin{minipage}[t]{0.4\textwidth} % 45% of the page width for name
	\vspace{-\baselineskip} % Required for vertically aligning minipages
	
	\colorbox{black}{{\HUGE\textcolor{white}{\textbf{\MakeUppercase{Ryan}}}}} % First name
	
	\colorbox{black}{{\HUGE\textcolor{white}{\textbf{\MakeUppercase{Walker}}}}} % Last name
	
	\vspace{6pt}
	
	{\huge Roboticist} % Career or current job title
\end{minipage}
\begin{minipage}[t]{0.275\textwidth} % 27.5% of the page width for the first row of icons
	\vspace{-\baselineskip} % Required for vertically aligning minipages
	
	\icon{MapMarker}{12}{Vancouver, BC}\\
	\icon{Phone}{12}{+1 604 612 9272}\\
	\icon{At}{12}{\href{mailto:ryan.cjw@gmail.com}{ryan.cjw@gmail.com}}\\	
\end{minipage}
\begin{minipage}[t]{0.275\textwidth} % 27.5% of the page width for the second row of icons
	\vspace{-\baselineskip} % Required for vertically aligning minipages
	
	\icon{Github}{12}{\href{https://github.com/machine-hum}{github.com/machine-hum}}\\
\end{minipage}

\vspace{0.5cm}

%----------------------------------------------------------------------------------------
%	INTRODUCTION, SKILLS AND TECHNOLOGIES
%----------------------------------------------------------------------------------------

\cvsect{Who Am I?}

\begin{minipage}[t]{0.4\textwidth} % 40% of the page width for the introduction text
	\vspace{-\baselineskip} % Required for vertically aligning minipages

I'm a mission driven engineer striving to build something better than what we
have. For the last eight years I've been exercising my skills in hardware,
software and robotics. My passions lie in Open Source tech, Embedded Systems,
Linux, robots and general tinkering.  

\end{minipage}
\hfill % Whitespace between
\begin{minipage}[t]{0.5\textwidth} % 50% of the page for the skills bar chart
	\vspace{-\baselineskip} % Required for vertically aligning minipages
	\begin{barchart}{5.5}
		\baritem{Electrical Design}{90}
		\baritem{Firmware (C/C++)}{100}
		\baritem{Python/Data Science}{60}
		\baritem{Robotics}{85}
		\baritem{Linux}{70}
	\end{barchart}
\end{minipage}

%----------------------------------------------------------------------------------------
%	EXPERIENCE
%----------------------------------------------------------------------------------------
\cvsect{Experience}

\begin{entrylist}
	\entry
		{2020 -- Present}
		{Firmware Engineer}
		{Facebook, Oculus Devision}
        {
        At Oculus I develop firmware in the OS and kernel. I'm familiar with
        Android/Linux development as well as using bare metal RTOS's like
        FreeRTOS and zephyr.
        }
	\entry
		{2015 -- 2020}
		{Roboticist}
		{MistyWest}
        {
        MistyWest is a full stack research and engineering house. Up to this
        point I've worked on aquatic life saving devices, polar bear trackers
        for the WWF, partial sensors, research devices that run current through
        brains, magnetic field generators, medical devices, fridges to be used
        in developing countries and several other projects I can't talk about.
        I've responsible for PCB design and layout, firmware development, embedded
        Linux development and robotics.
        }
    \entry
		{2014 -- 2015}
		{Control Systems Designer}
		{MPM Engineering}
		{
        MPM Engineering develops 3D scanning systems for lumber automation. My
        role was to design control systems which enabled the automation of
        multi-ton machinery.  
        }
	\entry
		{2013 -- 2014}
		{Electromechanical Technologist}
		{Medico Supplies}
		{
        Medico designs 3D scanners and CNC machines for use in the prosthetics
        industry. I was responsible for designing and testing the embedded
        compute inside the scanners. I additionally designed the control system
        for the CNC machines sold.
        }
\end{entrylist}

%----------------------------------------------------------------------------------------
%	EDUCATION
%----------------------------------------------------------------------------------------

\cvsect{Education}

\begin{entrylist}
	\entry
		{2014 -- 1/2020\\ \footnotesize{part time}}
		{Degree in Electronics Engineering}
		{British Columbia Institute of Technology}
		{This degree focuses on advanced electronics and specifically their role in control systems, digital signal processing, filters, and real-time systems. My capstone project is a device for tracking the migration of birds. It uses an STM32F7 and TensorFlow lite to speciate birds with acoustic data. From there, a timeseries heatmap is generated to understand bird hotspots.}
	\entry
		{2011 -- 2013}
		{Diploma in Robotics and Mechatronics}
		{British Columbia Institute of Technology}
		{This diploma focused on building a solid foundation in electronics, math and physics before moving into higher level classes, such as robotic application, microcontroller theory and sensor theory. My final project was to design and build a delta robot for use in a pick and place application. This project was sponsored by a local robotics company.}
\end{entrylist}

%----------------------------------------------------------------------------------------
% Open Source Work	
%----------------------------------------------------------------------------------------

\cvsect{Open Source Work}

\begin{entrylist}
	\entry
		{}
		{Modular Synth Work}
		{}
		{We design open source modular synths. A modular synth is a synthesizer which resided on a panel with several other synths of same format. They can then be linked together with patch cables to make various sounds.}
	\entry
		{}
		{Worlds: A distributed MMO}
		{worldsmmo.com}
		{Worlds is a protocol which manages the large scale economic components of a distributed MMO. This protocol enables fair scaling of a universe based on games developed by independent parties.}

\end{entrylist}
\end{document}
